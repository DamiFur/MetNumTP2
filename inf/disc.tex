\section{Discusi\'on}

\subsection{kNN}

Primero, experimentamos utilizando los tests provistos por la catedra (test1.in y test2.in) con los valores que ya ten\'ia para ver que resultado obtenemos.

\begin{table}[H]
\centering
\begin{tabular}{| l | c | c | c | c | c |}
\hline
\#N\'umero de partici\'on & precision & recall & f1-score & tiempo en nanosegundos & tiempo en minutos \\
\hline

0 & 0.966017 & 0.96505 & 0.965309 & 321379808901 & 5.35 \\
1 & 0.972961 & 0.971499 & 0.972077 & 314427110608 & 5.24 \\
2 & 0.966673 & 0.965729 & 0.965946 & 313701417004 & 5.22 \\
3 & 0.968304 & 0.967631 & 0.967826 & 315968779893 & 5.26 \\
4 & 0.964213 & 0.963188 & 0.96337 & 300517782527 & 5 \\
5 & 0.967247 & 0.966019 & 0.96635 & 300517782527 & 5 \\
6 & 0.966298 & 0.965881 & 0.965841 & 303090224951 & 5.05 \\
7 & 0.973695 & 0.972808 & 0.973107 & 318840910396 & 5.31 \\
8 & 0.970612 & 0.969215 & 0.969717 & 319504538983 & 5.32 \\
9 & 0.966593 & 0.966049 & 0.966168 & 324889276456 & 5.31 \\

\hline
\end{tabular}
\caption{Experimentaci\'on kNN con test1 provisto por la catedra}
\end{table}

\begin{table}[H]
\centering
\begin{tabular}{| l | c | c | c | c | c |}
\hline
\#N\'umero de partici\'on & precision & recall & f1-score & tiempo en nanosegundos & tiempo en minutos \\
\hline

0 & 0.969592 & 0.968373 & 0.968838 & 599496056343 & 9.99 \\
1 & 0.965361 & 0.964633 & 0.964736 & 584128017663 & 9.73 \\
2 & 0.965809 & 0.964929 & 0.96514 & 579415970721 & 9.65 \\
3 & 0.965984 & 0.964777 & 0.965126 & 551468697663 & 9.19 \\
4 & 0.968909 & 0.967957 & 0.968287 & 572571771131 & 9.54 \\

\hline
\end{tabular}
\caption{Experimentaci\'on kNN con test2 provisto por la catedra}
\end{table}