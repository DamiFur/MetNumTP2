\section{Discusi\'on}

\subsection{kNN}

Primero, experimentamos utilizando los tests provistos por la catedra (test1.in y test2.in) con los valores que ya ten\'ia para ver que resultado obtenemos.

\subsubsection{Experimentaci\'on con test1}

Primero, experimentamos con el archivo tal cual fue provisto por la catedra:

\begin{table}[H]
\centering
\begin{tabular}{| l | c | c | c | c | c |}
\hline
N\'umero de partici\'on & precision & recall & f1-score & tiempo en minutos \\
\hline

0 & 0.968306 & 0.967428 & 0.967672 & 5 \\
1 & 0.972546 & 0.971092 & 0.971666 & 5 \\
2 & 0.967754 & 0.966805 & 0.967054 & 5 \\
3 & 0.969186 & 0.968603 & 0.968779 & 5 \\
4 & 0.964399 & 0.963591 & 0.963692 & 5 \\
5 & 0.968841 & 0.967744 & 0.968078 & 5 \\
6 & 0.968534 & 0.968088 & 0.968098 & 5 \\
7 & 0.975505 & 0.974413 & 0.974799 & 5 \\
8 & 0.972179 & 0.970912 & 0.971412 & 5 \\
9 & 0.968099 & 0.967568 & 0.967727 & 5 \\

\hline
\end{tabular}
\caption{Experimentaci\'on kNN con $k=5$}
\end{table}

Luego probamos aumentar un poco el $k$, llegando a 10, y notamos que para las tres primeras particiones no variaba mucho el resultado y daba un poco menos:

\begin{table}[H]
\centering
\begin{tabular}{| l | c | c | c | c | c |}
\hline
N\'umero de partici\'on & precision & recall & f1-score & tiempo en minutos \\
\hline

0 & 0.966002 & 0.964818 & 0.965094 & 5 \\
1 & 0.968972 & 0.967024 & 0.967752 & 5 \\
2 & 0.967238 & 0.965855 & 0.966201 & 5 \\

\hline
\end{tabular}
\caption{Experimentaci\'on kNN con $k=10$}
\end{table}

Decidimos luego probar con un $k$ que sea mas grande en orden de magnitud, y elegimos 100.  Los resultados con los metodos de medici\'on usados bajaron mas (si bien siguen siendo variaciones del orden de $0.0x$):

\begin{table}[H]
\centering
\begin{tabular}{| l | c | c | c | c | c |}
\hline
N\'umero de partici\'on & precision & recall & f1-score & tiempo en minutos \\
\hline

0 & 0.935468 & 0.929896 & 0.930686 & 5 \\
1 & 0.946788 & 0.94221 & 0.943543 & 5 \\
2 & 0.937593 & 0.931579 & 0.93277 & 5 \\
3 & 0.936482 & 0.931388 & 0.932308 & 5 \\
4 & 0.93057 & 0.926095 & 0.926405 & 5 \\
5 & 0.936951 & 0.931295 & 0.932383 & 5 \\
6 & 0.936962 & 0.931634 & 0.932728 & 5 \\
7 & 0.943187 & 0.936894 & 0.938434 & 5 \\
8 & 0.934357 & 0.9274 & 0.929423 & 5 \\
9 & 0.934152 & 0.929181 & 0.930415 & 5 \\

\hline
\end{tabular}
\caption{Experimentaci\'on kNN con $k=100$}
\end{table}

Con estos resultados, nos llevan a pensar que con $k$ mas grandes que el 5 estamos obteniendo resultados peores, asi que veremos que sucede con $k$ mas chicos. En las siguientes tablas, mostramos en las primeras 3 particiones que ocurre en este caso:

\begin{table}[H]
\centering
\begin{tabular}{| l | c | c | c | c | c |}
\hline
N\'umero de partici\'on & precision & recall & f1-score & tiempo en minutos \\
\hline

0 & 0.970793 & 0.970082 & 0.970253 & 5 \\
1 & 0.974946 & 0.973822 & 0.974283 & 4 \\
2 & 0.968389 & 0.967444 & 0.967707 & 5 \\

\hline
\end{tabular}
\caption{Experimentaci\'on kNN con $k=4$}
\end{table}

Vemos una mejora con respecto a $k=5$, que en la tercer partici\'on es pequen\~na pero en el resto es un poco mejor.

\begin{table}[H]
\centering
\begin{tabular}{| l | c | c | c | c | c |}
\hline
N\'umero de partici\'on & precision & recall & f1-score & tiempo en minutos \\
\hline

0 & 0.969364 & 0.968575 & 0.968781 & 5 \\
1 & 0.9753 & 0.974382 & 0.974759 & 5 \\
2 & 0.968136 & 0.96754 & 0.967641 & 5 \\

\hline
\end{tabular}
\caption{Experimentaci\'on kNN con $k=3$}
\end{table}

Si comparamos con $k=4$, podemos decir que en general es similar salvo la primera partici\'on que da un poco menos.

\begin{table}[H]
\centering
\begin{tabular}{| l | c | c | c | c | c |}
\hline
N\'umero de partici\'on & precision & recall & f1-score & tiempo en minutos \\
\hline

0 & 0.967597 & 0.967088 & 0.967186 & 5 \\
1 & 0.969326 & 0.968344 & 0.968751 & 5 \\
2 & 0.967231 & 0.966429 & 0.966642 & 5 \\

\hline
\end{tabular}
\caption{Experimentaci\'on kNN con $k=2$}
\end{table}


\begin{table}[H]
\centering
\begin{tabular}{| l | c | c | c | c | c |}
\hline
N\'umero de partici\'on & precision & recall & f1-score & tiempo en minutos \\
\hline

0 & 0.967597 & 0.967088 & 0.967186 & 5 \\
1 & 0.969326 & 0.968344,0.968751 & 5 \\
2 & 0.967231 & 0.966429,0.966642 & 5 \\

\hline
\end{tabular}
\caption{Experimentaci\'on kNN con $k=1$}
\end{table}


\subsubsection{Experimentaci\'on con test2}

Al igual que el otro test, comenzamos con el k tal cual fue provisto por la catedra con $k = 3$:
\begin{table}[H]
\centering
\begin{tabular}{| l | c | c | c | c | c |}
\hline
N\'umero de partici\'on & precision & recall & f1-score & tiempo en minutos \\
\hline

0 & 0.971022 & 0.969965 & 0.970386 & 9 \\
1 & 0.966954 & 0.966284 & 0.966407 & 9 \\
2 & 0.966228 & 0.965548 & 0.965694 & 9 \\
3 & 0.968304 & 0.967198 & 0.967568 & 9 \\
4 & 0.96997 & 0.969218 & 0.969475 & 9 \\

\hline
\end{tabular}
\caption{Experimentaci\'on kNN con $k=3$}

\end{table}