\section{Introducción}

Durante el presente trabajo pr\'actico buscaremos analizar distintas t\'ecnicas de ''machine learning'' vistas en clase aplicadas a un problema puntual. En nuestro caso, tendremos imagenes que representan alg\'un d\'igito, y queremos saber cu\'al es. Para eso, tendremos casos de ''training'' en donde cada imagen tendra un determinado ''label'' representando a qu\'e d\'igito corresponde. Con toda esa informaci\'on de entrenamiento, tendremos que poder decir que n\'umero representa las distintas imagenes dentro de ''test''.\\

Para poder analizar nuestros algoritmos, podemos particionar ''training'' y tener algunas imagenes que usaremos para saber si nuestro algoritmo puede decir qu\'e d\'igito es. De esta manera, podemos comparar contra su ''label'' (que ya sabemos) e ir probando con particiones de distintos tama\~nos que sucede. Notemos que cada imagen esta representada por una tira de 783 p\'ixeles.\\

Utilizaremos los m\'etodos de $kNN$, $PCA + kNN$ y $PLS-DA + kNN$ que fueron vistos en clase, para poder predecir qu\'e d\'igito es cada una. En la secci\'on de desarrollo, explicaremos la implementaci\'on de cada uno en nuestro programa. En resultados, mostraremos lo obtenido en las distantas experimentaciones utilizando distintas formas de medir una experimentaci\'on. En desarrollo hablaremos sobre las dos elegidas y su implementaci\'on. En discusi\'on, expondremos un an\'alisis sobre los resultados obtenidos y buscaremos comparar los distintos metodos entre si. En conclusi\'on, le daremos un cierre al trabajo pr\'actico con lo que hayamos sacado en limpio y aprendido.\\

Nuestro programa recibe ciertos parametros y formatos de archivo que est\'an explicados en el enunciado del TP, igualmente, buscaremos aclarar en cada resultado cuales fueron los parametros de nuestro programa para que pueda ser reproducido el experimento en otro lado.